\documentclass[10pt]{report}
%\usepackage{ae} % or {zefonts}
\usepackage[T1]{fontenc}
%\usepackage[utf8]{inputenc}
\usepackage[english]{babel}
\usepackage{makeidx}
\usepackage{amsmath}
\usepackage{amsfonts}
\usepackage{multicol}
\usepackage{float}
\usepackage[small]{caption}
\usepackage{textcomp}

\usepackage{graphicx}
%\usepackage{color}
\usepackage[colorlinks]{hyperref}
%\usepackage{epsfig}
\usepackage[a4paper,margin=2cm]{geometry}

% Get it from CTAN if it is not included in your TEX distribution
%\usepackage{europs}
%\DeclareInputText{128}{\EUR} % ANSI code for euro: �
%   \EURhv   selects EuroSans
%   \EURtm   selects EuroSerif
%   \EURcr   selects EuroMono
%   \EUR    selects one of the three above, depending on the current
%        context
%
%   \EURofc  selects EuroSans Regular independent of context
%        N.B.: This is the only "official" Euro symbol. If you
%           want to conform with the rules of the EU (or
%           whoever), you may only use this symbol.
%
%\usepackage{eurosym}
%\DeclareInputText{128}{\euro} % ANSI code for euro: �

%\usepackage{lscape}
\usepackage{algorithm}% http://ctan.org/pkg/algorithm
\usepackage{algpseudocode}% http://ctan.org/pkg/algorithmicx
\makeatletter
\newlength{\trianglerightwidth}
\settowidth{\trianglerightwidth}{$\triangleright$~}
\algnewcommand{\LineComment}[1]{\Statex \hskip\ALG@thistlm $\triangleright$ #1}
\algnewcommand{\LineCommentCont}[1]{\Statex \hskip\ALG@thistlm%
  \parbox[t]{\dimexpr\linewidth-\ALG@thistlm}{\hangindent=\trianglerightwidth \hangafter=1 \strut$\triangleright$ #1\strut}}
\makeatother

\title{Internship report}
\author{Tamar Huygen \\ tamar@huygen.nl}

\begin{document}
\maketitle

\begin{abstract}
  Participation iGEM 2015 my contribution
\end{abstract}

\chapter{Introduction}
iGem cyanobacteria, filipe overview literature,
oview project, stability, my contribution
When is something considered stable?
Why a symbiose?
Why Synechocystis?
Why E. Coli?
Why the algorithms?
Why kinetic models?
How does this module fit into the project?

\chapter{Algorithms}

\begin{abstract}
  
\end{abstract}

\section{Introduction}
An important goal in the bio-industry is the production of certain carbon containing compounds. When we let micro-organisms produce these compounds, there is always a stability challenge. Usually it is better in terms of fitness for an organism use all carbon it can for the generation of biomass. So consider an organism which where we have inserted genes which forces the organism to produce a certain compound and transport it out of the cell, so that we may use it. A mutation may yield an organism which does no longer produce the compound. Organisms with this mutation likely have a fitness advantage over organisms without this mutation, because they can use all carbon compounds for generation of biomass and don't leak carbon compound to the environment. This means that soon after the appearance of the mutation the non-productive organisms will take over the entire population and the production stops and has to be re-initiated. We want to address this problem and have created an algorithm which searches for solutions that force an organism to produce a carbon compound, with a much smaller chance of rendering a non-productive population.

\subsection{Models}\label{sec:intro_models}
FBA models consist of a set of differential equations which model the flux of mass through reactions. It is important to note that the base of the model consists of reactions. Genes play an important role in the models, but only in terms of the influence of the genes on the reactions. In reality there is a complex interaction between genes, gene regulation, transcription, translation and the reactions. In our models the genes are only encoded in the influence they eventually have on the flux through the reactions. In the models we use the reactions sometimes have gene associations. These gene associations list genes which when turned on, make the reaction possible. If there are multiple genes listed in the gene association they can have an or- or an and- relationship. In figure !! two reactions are shown with the genes that are listed in the gene association of these reactions. 

\subsection{Stable compound generator}
The goal of the stable compound generator is to find genes which can be deleted in order to create an organism which has to produce a certain carbon compound. Consider the reaction scheme in figure \ref{fig:reaction_overview}. Our goal is to produce compound B. Our initial plan to achieve this goal is to take out reaction 2. Assume reaction 2 is the only reaction which consumes B. By taking out the reaction we hope the cell will not be able to use B and accumulate and transport it out of the cell.

We have 2 prerequisites to be able to have a stable production of compound B. The first is that reaction 1 is essential for the production of biomass. If cells that don't use reaction 1 are still viable and produce biomass, they will most likely have a fitness advantage over cells that still use reaction 1, because they won't have to produce compound B and can use more carbon for biomass generation. But when cells that take out reaction 1 are non-viable the cells can never take over the population.
The second requirement is that we are able to take out reaction 2 and still yield viable cells, so reaction 2 has to be non-essential for the production of biomass. It is reaction 2 which we want to find. So a first algorithm can look for all reactions that consume B and take all of them out and see if the cells still produce biomass. If so, it returns reaction 2 to us.

\subsubsection{Combinatorial problems}
Now consider figure !! Here normally B is consumed by reaction 2 and reaction 3 is normally not used by the cell.
Only when reaction 2 is taken out, reaction 3 is used to consume B.
So we have to take out reaction 2 and reaction 3 to be able to produce B.
The algorithm we just thought of would return reaction 2 to us, because when it looks for reactions that consume B it will only find reaction 2 because initially the flux through reaction 3 was equal to 0.
To solve this problem we have to look for all reactions which are able to consume B.
But this is where it gets complex and where we already can see the scheme is a simplification.
Most reactions in a cell are chemical balances which are driven in a certain direction because the products are consumed again, driving a mass flux through the reaction.
But potentially almost all reactions in the model can potentially also happen in the reverse direction when the product is no longer consumed.
This means that when we look for reactions that can potentially consume B, we also find reaction 1.
We don't want to take out reaction 1 because we have earlier assumed it was essential for the production of biomass and will yield non-viable cells.
So an algorithm which looks for all reactions that potentially can consume B and takes them all down will not find anything in this situation.
Another possibility is that there are multiple reactions such as reaction 1 which produce B, and are essential for biomass generation.
So we cannot just find reaction 1 and take out all other reactions that potentially consume B.
So the algorithm has to list all reactions that potentially consume B, and find a subset of reactions which it can take out such that the cell is still viable and produces B.
In order to find this subset we have to try out all possible subsets. In this scheme the possibilities the algorithm would create the following list of reactions: ${1, 2, 3}$.
Then it will have to create a list of all combinations of these reactions: ${{1},{2},{3},{1,2},{1,3},{2,3}}$ (the list doesn't contain $\emptyset$, nor ${1,2,3}$ because we don't have to try taking out no reactions, because then we wouldn't change the cell and we don't have to try taking out all reactions, because then we will also take out all reactions that produce B and we cannot produce B to begin with).
The algorithm then has to try each of these subsets.
In this case that is doable.
In all combinations where reaction 1 is taken out the cells are not viable, so the algorithm won't return them and in cases where only reaction 2 or reaction 3 is taken out, B won't be produced, so the algorithm won't return them, but the algorithm will return ${2,3}$ given that reactions 2 and 3 are not essential for cell growth.
However if $n$ is the number of reactions where B is involved in, the amount of possible combinations increases with $2^n$.
So the algorithm will exponentially become slower, the more reactions are involved.
In the implementation of the algorithm we will define an upper limit of combinations which are tried out before it stops. We will start with combinations in which only a small number of reactions has to be taken out, because this is also the easiest to achieve in real live.

\subsubsection{Completeness}
Now consider the reaction scheme shown in figure !!. To produce B we might take out reaction 2, but another possibility might be to take out reaction 3. When C accumulates the flux through reaction 2 might become 0 and B might also be produced and transported out of the cell. Our algorithm won't find this possibility, because it only searches for reactions that potentially consume B. This means our algorithm is not complete and will not return all possible reactions which can be taken out in order to create an organism that produces B. We have shortly tried to find reactions further down the pathways, but usually this yields almost all reactions of a cell. This in combination with the earlier described combinatorial problems yields a unusable algorithm. So for now we accept that the algorithm is not complete and does not guarantee that it finds all possible reactions that can be taken out to produce B.

\subsubsection{From reactions to genes}
Up until now we've only spoken about cells in terms of reactions and our algorithm takes out reactions. In reality it is very hard to influence reactions in cells directly. More often we try to manipulate reactions that happen in cells through genes. We can stop reactions for example by taking out genes that encode enzymes which catalyze a reaction. Since we want to find out what genes we can take out instead of reactions directly, we also would want our algorithm to provide these genes for us. We can change the algorithm to do this.
The algorithm than has to do the following:
\begin{itemize}
\item List all reactions that potentially consume B
\item List all combinations of reactions that when taken out potentially can make the cell produce B
\item For each combination of reactions just listed, for each reaction in the combination, list the genes associated with the reaction.
\item For each listed gene association list the combination of genes that can be turned of in order to make the reaction no longer happen.
\item For each combination of genes just listed, turn off the genes (this means the algorithm will find each reaction which has the gene in its gene association and determines which of these reactions is taken out when the genes are taken out of this genes and sets the flux through these reactions to 0)
\item Analyze model and check if cell still produces biomass and if compound B is produced.
\item return successful combinations of genes.
\end{itemize}


\begin{figure}[hbtp]
  \centering
     \input{reaction_overview.pdftex_t}
      \caption{Scematic overview of a pathway in which product C is produced when reaction 3 is taken out. The algorithm finds reaction 3 and takes it out and checks if biomass is still produced and if so if C is produced. However, suppose there is another pathway which the cell can use to recycle substance C, (in this figure symbolized by reaction 5), then the cell might make use of this pathway to recycle C. Taking out all reactions which might use substance C will also take out reaction 2, since in reality most of the reactions are . When reaction 2 is taken out, C won't be produced.}
  \label{fig:reaction-overview}
\end{figure}



\section{Methods}
\subsection{Implementation}
\subsubsection{Genes}
Because the models we use are not perfect, the and- and or- relations between the genes cannot be trusted. We still want the algorithm to return genes and so we have a different implementation than the one given in the introduction.

\begin{algorithm}
  \caption{Old (simple) stable compound generator algorithm. We find reactions associated with a certain compound and take a combination of these reactions and switch them of. In reality we cannot switch off reactions, we have to remove genes. This is why this is just a simple version of the algorithm.}\label{alg:scg_old}
  \begin{algorithmic}[1]
    \Procedure{FindReactions}{$compound$, $model$}
        \State $hits \gets \text{\{ \}}$
        \State $r1 \gets [\text{ }]$ 
        \For{$r \in model.reactions$}
        \LineCommentCont{$model.reactions$ gives a list of all reactions in the model. $r$ loops over all these reactions}
            \If{$compound \in r.products \vee compound \in r.reactans$}
            \LineCommentCont{$r.products$ is a list of all products produced by reaction and $r.ractants$ is a list of all reactants of reaction $r$.}
                \State $r1.append(r)$
            \EndIf
        \EndFor
        \If{$length(r1)>1$}
            \State $combinations \gets [comb(r1)]$
            \LineCommentCont{$combinations$ is a list of lists. Each list in combinations contains a combination of reactions that are in r1}
            \State $genes \gets \text{\{ \}}$
            \For{$c \in combinations$}
                \For{$r \in c$}
                    \State $r.flux \gets 0$
                \EndFor
                \State $analyseModel(model)$
                \If{$growth > 0$}
                    \If{$compound \in hits.keys()$}
                        \State $hits[compound].append(c)$
                        \Else{$hits[compound] \gets [c]$}
                    \EndIf
                \EndIf
            \EndFor
        \EndIf
    \EndProcedure
  \end{algorithmic}
\end{algorithm}



\section{Results}
\subsection{Stable compound generator}
het algoritme (een paar relevante algoritmes)
toepassing lijst met genen die mogelijk uit te schakelen zijn
Toepassing van iGem

\section{Discussion}
completeness
sinks
Christine's experiment
opbrengst

\chapter{Kinetic models}

\begin{abstract}
  
\end{abstract}

\section{Introduction}
Why do we want to use kinetic models?
Under what restrictions does a symbiosis converge.
When is something considered stable?

\section{Methods}
type of models
sets of differential equations
software

\section{Results}
convergenge
pictures
toepassing (Stijn en Hugo)
stability analysis

\section{Discussion}
Different values than Stijn's values
maybe add delay to get oscillations in future research
too much combinations of parameters
too much parameters

\chapter{Discussion}
Results in the light of the iGEM project
Why a 2 organisms, why symbiose, complications with 2 organisms in a tank
How to advance.


\chapter{Acknowledgement}

\end{document}
