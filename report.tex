\documentclass[10pt]{report}
%\usepackage{ae} % or {zefonts}
\usepackage[T1]{fontenc}
%\usepackage[utf8]{inputenc}
\usepackage[english]{babel}
\usepackage{makeidx}
\usepackage{amsmath}
\usepackage{amsfonts}
\usepackage{multicol}
\usepackage{float}
\usepackage[small]{caption}
\usepackage{textcomp}

\usepackage{graphicx}
%\usepackage{color}
\usepackage[colorlinks]{hyperref}
%\usepackage{epsfig}
\usepackage[a4paper,margin=2cm]{geometry}

% Get it from CTAN if it is not included in your TEX distribution
%\usepackage{europs}
%\DeclareInputText{128}{\EUR} % ANSI code for euro: �
%   \EURhv   selects EuroSans
%   \EURtm   selects EuroSerif
%   \EURcr   selects EuroMono
%   \EUR    selects one of the three above, depending on the current
%        context
%
%   \EURofc  selects EuroSans Regular independent of context
%        N.B.: This is the only "official" Euro symbol. If you
%           want to conform with the rules of the EU (or
%           whoever), you may only use this symbol.
%
%\usepackage{eurosym}
%\DeclareInputText{128}{\euro} % ANSI code for euro: �

%\usepackage{lscape}
\usepackage{algorithm}% http://ctan.org/pkg/algorithm
\usepackage{algpseudocode}% http://ctan.org/pkg/algorithmicx
\makeatletter
\newlength{\trianglerightwidth}
\settowidth{\trianglerightwidth}{$\triangleright$~}
\algnewcommand{\LineComment}[1]{\Statex \hskip\ALG@thistlm $\triangleright$ #1}
\algnewcommand{\LineCommentCont}[1]{\Statex \hskip\ALG@thistlm%
  \parbox[t]{\dimexpr\linewidth-\ALG@thistlm}{\hangindent=\trianglerightwidth \hangafter=1 \strut$\triangleright$ #1\strut}}
\makeatother

\title{Internship report}
\author{Tamar Huygen \\ tamar@huygen.nl}

\begin{document}
\maketitle

\begin{abstract}
  Participation iGEM 2015 my contribution
\end{abstract}

\chapter{Introduction}
iGem cyanobacteria, filipe overview literature,
oview project, stability, my contribution
When is something considered stable?
Why a symbiose?
Why Synechocystis?
Why E. Coli?
Why the algorithms?
Why kinetic models?
How does this module fit into the project?

\chapter{Algorithms}

\begin{abstract}
  
\end{abstract}

\section{Introduction}
2 problemen
plaatjes
type modellen waar algorithmes
FBA models
problem of the sinks
auxotrophes
problem of combinations
compleetheid


\begin{figure}[hbtp]
  \centering
     \input{reaction_overview.pdftex_t}
      \caption{Reaction-overview}
  \label{fig:reaction-overview}
\end{figure}



\section{Methods}
approach
pysces cbm
input en output types
toepassing: in een loop modellen gerund


\begin{algorithm}
  \caption{Old (simple) stable compound generator algorithm. We find reactions associated with a certain compound and take a combination of these reactions and switch them of. In reality we cannot switch off reactions, we have to remove genes. This is why this is just a simple version of the algorithm.}\label{alg:scg_old}
  \begin{algorithmic}[1]
    \Procedure{FindReactions}{$compound$, $model$}
        \State $hits \gets \text{\{ \}}$
        \State $r1 \gets [\text{ }]$ 
        \For{$r \in model.reactions$}
        \LineCommentCont{$model.reactions$ gives a list of all reactions in the model. $r$ loops over all these reactions}
            \If{$compound \in r.products \vee compound \in r.reactans$}
            \LineCommentCont{$r.products$ is a list of all products produced by reaction and $r.ractants$ is a list of all reactants of reaction $r$.}
                \State $r1.append(r)$
            \EndIf
        \EndFor
        \If{$length(r1)>1$}
            \State $combinations \gets [comb(r1)]$
            \LineCommentCont{$combinations$ is a list of lists. Each list in combinations contains a combination of reactions that are in r1}
            \State $genes \gets \text{\{ \}}$
            \For{$c \in combinations$}
                \For{$r \in c$}
                    \State $r.flux \gets 0$
                \EndFor
                \State $analyseModel(model)$
                \If{$growth > 0$}
                    \If{$compound \in hits.keys()$}
                        \State $hits[compound].append(c)$
                        \Else{$hits[compound] \gets [c]$}
                    \EndIf
                \EndIf
            \EndFor
        \EndIf
    \EndProcedure
  \end{algorithmic}
\end{algorithm}



\section{Results}
\subsection{Stable compound generator}
het algoritme (een paar relevante algoritmes)
toepassing lijst met genen die mogelijk uit te schakelen zijn
Toepassing van iGem

\section{Discussion}
completeness
sinks
Christine's experiment
opbrengst

\chapter{Kinetic models}

\begin{abstract}
  
\end{abstract}

\section{Introduction}
Why do we want to use kinetic models?
Under what restrictions does a symbiosis converge.
When is something considered stable?

\section{Methods}
type of models
sets of differential equations
software

\section{Results}
convergenge
pictures
toepassing (Stijn en Hugo)
stability analysis

\section{Discussion}
Different values than Stijn's values
maybe add delay to get oscillations in future research
too much combinations of parameters
too much parameters

\chapter{Discussion}
Results in the light of the iGEM project
Why a 2 organisms, why symbiose, complications with 2 organisms in a tank
How to advance.


\chapter{Acknowledgement}

\end{document}
