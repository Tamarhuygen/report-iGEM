\documentclass[10pt]{report}
%\usepackage{ae} % or {zefonts}
\usepackage[T1]{fontenc}
%\usepackage[utf8]{inputenc}
\usepackage[english]{babel}
\usepackage{makeidx}
\usepackage{amsmath}
\usepackage{amsfonts}
\usepackage{multicol}
\usepackage{float}
\usepackage[small]{caption}
\usepackage{textcomp}

\usepackage{graphicx}
%\usepackage{color}
\usepackage[colorlinks]{hyperref}
%\usepackage{epsfig}
\usepackage[a4paper,margin=2cm]{geometry}

% Get it from CTAN if it is not included in your TEX distribution
%\usepackage{europs}
%\DeclareInputText{128}{\EUR} % ANSI code for euro: �
%   \EURhv   selects EuroSans
%   \EURtm   selects EuroSerif
%   \EURcr   selects EuroMono
%   \EUR    selects one of the three above, depending on the current
%        context
%
%   \EURofc  selects EuroSans Regular independent of context
%        N.B.: This is the only "official" Euro symbol. If you
%           want to conform with the rules of the EU (or
%           whoever), you may only use this symbol.
%
%\usepackage{eurosym}
%\DeclareInputText{128}{\euro} % ANSI code for euro: �

%\usepackage{lscape}


\title{Internship report}
\author{Tamar Huygen \\ tamar@huygen.nl}

\begin{document}
\maketitle

\begin{abstract}
  Participation iGEM 2015 my contribution
\end{abstract}

\chapter{Introduction}
iGem cyanobacterien filipe overview literature, oview project, stability, my contribution



\chapter{Algorithms}

\begin{abstract}
  
\end{abstract}

\section{Introduction}
2 problemen
plaatjes
type modellen waar algorithmes
FBA models
problem of the sinks
auxotrophes
problem of combinations
compleetheid

\section{Methods}
approach
pysces cbm
input en output types
toepassing: in een loop modellen gerund

\section{Results}
\subsection{Stable compound generator}
het algoritme (een paar relevante algoritmes)
toepassing lijst met genen die mogelijk uit te schakelen zijn
Toepassing van iGem

\section{Discussion}
completeness
sinks
Christine's experiment
opbrengst

\chapter{Kinetic models}

\begin{abstract}
  
\end{abstract}

\section{Introduction}
Why do we want to use kinetic models?
Under what restrictions does a symbiosis converge.
When is something considered stable?

\section{Methods}
type of models
sets of differential equations
software

\section{Results}
convergenge
pictures
toepassing (Stijn en Hugo)
stability analysis

\section{Discussion}
Different values than Stijn's values
maybe add delay to get oscillations in future research
too much combinations of parameters
too much parameters

\chapter{Discussion}
Results in the light of the iGEM project
Why a 2 organisms, why symbiose, complications with 2 organisms in a tank 
How to advance

\chapter{Acknowledgement}

\end{document}
