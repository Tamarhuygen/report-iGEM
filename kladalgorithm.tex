\documentclass{article}
\usepackage{amsmath}
\usepackage{algorithm}% http://ctan.org/pkg/algorithm
\usepackage{algpseudocode}% http://ctan.org/pkg/algorithmicx
\makeatletter
\newlength{\trianglerightwidth}
\settowidth{\trianglerightwidth}{$\triangleright$~}
\algnewcommand{\LineComment}[1]{\Statex \hskip\ALG@thistlm $\triangleright$ #1}
\algnewcommand{\LineCommentCont}[1]{\Statex \hskip\ALG@thistlm%
  \parbox[t]{\dimexpr\linewidth-\ALG@thistlm}{\hangindent=\trianglerightwidth \hangafter=1 \strut$\triangleright$ #1\strut}}
\makeatother
\begin{document}
\begin{algorithm}
  \caption{Old (simple) stable compound generator algorithm. We find reactions associated with a certain compound and take a combination of these reactions and switch them of. In reality we cannot switch off reactions, we have to remove genes. This is why this is just a simple version of the algorithm. }\label{alg:scg_old}
  \begin{algorithmic}[1]
    \Procedure{FindReactions}{$compound$, $model$}
        \State $hits \gets \text{\{ \}}$
        \State $r1 \gets [\text{ }]$ 
        \For{$r \in model.reactions$}
        \LineCommentCont{$model.reactions$ gives a list of all reactions in the model. $r$ loops over all these reactions}
            \If{$compound \in r.products \vee compound \in r.reactans$}
            \LineCommentCont{$r.products$ is a list of all products produced by reaction and $r.ractants$ is a list of all reactants of reaction $r$.}
                \State $r1.append(r)$
            \EndIf
        \EndFor
        \If{$length(r1)>1$}
            \State $combinations \gets [comb(r1)]$
            \LineCommentCont{$combinations$ is a list of lists. Each list in combinations contains a combination of reactions that are in r1}
            \State $genes \gets \text{\{ \}}$
            \For{$c \in combinations$}
                \For{$r \in c$}
                    \State $r.flux \gets 0$
                \EndFor
                \State $analyseModel(model)$
                \If{$growth > 0 \wedge compound produced$}
                    \If{$compound \in hits.keys()$}
                        \State $hits[compound].append(c)$
                        \Else{$hits[compound] \gets [c]$}
                    \EndIf
                \EndIf
            \EndFor
        \EndIf
    \EndProcedure
  \end{algorithmic}
\end{algorithm}
\end{document}